\section{Тест производительности}

\subsection{Условия тестирования}

Тестирование производительности проводилось на 100'000'000 строках, с ключами от 0 до 65535 и значениями от 0 до $2^{64} - 1$.
Время работы стандартной сортировки \texttt{std::stable\_sort} составило 47.5887 секунд, а время работы сортировки подсчётом --- 6.36868 секунд.
\begin{alltt}
discrete-labs git:(lab1) make test ARGS="./data/wide_range.txt output.txt"
Testing std::stable_sort...
Elapsed time: 47.5887 seconds
Testing countSort...
Elapsed time: 6.36868 seconds
\end{alltt}

Сортировка подсчётом выполнилась в 7.46 раз быстрее, чем стандартная сортировка. Это объясняется тем, что сортировка подсчётом \texttt{countSort} имеет линейную временную сложность $O(n + k)$, где $n$ — количество элементов, а $k$ — диапазон возможных значений ключей. В данном случае диапазон ключей ограничен (от 0 до 65535), что делает алгоритм особенно эффективным. При этом стандартная сортировка \texttt{std::stable\_sort}, обладающая временной сложностью $O(n \log n)$, требует больше времени для обработки больших объёмов данных и не учитывает специфику данных. Поэтому, такая разница в скорости является обоснованной и ожидаемой.

\pagebreak

